%%%%%%%%%%%%%%%%%%%%%%%%%%%%%%%%%%%%%%%
% Deedy - One Page Two Column Resume
% LaTeX Template
% Version 1.2 (16/9/2014)
%
% Original author:
% Debarghya Das (http://debarghyadas.com)
%
% Original repository:
% https://github.com/deedydas/Deedy-Resume
%
% IMPORTANT: THIS TEMPLATE NEEDS TO BE COMPILED WITH XeLaTeX
%
% This template uses several fonts not included with Windows/Linux by
% default. If you get compilation errors saying a font is missing, find the line
% on which the font is used and either change it to a font included with your
% operating system or comment the line out to use the default font.
%
%%%%%%%%%%%%%%%%%%%%%%%%%%%%%%%%%%%%%%
%
% TODO:
% 1. Integrate biber/bibtex for article citation under publications.
% 2. Figure out a smoother way for the document to flow onto the next page.
% 3. Add styling information for a "Projects/Hacks" section.
% 4. Add location/address information
% 5. Merge OpenFont and MacFonts as a single sty with options.
%
%%%%%%%%%%%%%%%%%%%%%%%%%%%%%%%%%%%%%%
%
% CHANGELOG:
% v1.1:
% 1. Fixed several compilation bugs with \renewcommand
% 2. Got Open-source fonts (Windows/Linux support)
% 3. Added Last Updated
% 4. Move Title styling into .sty
% 5. Commented .sty file.
%
%%%%%%%%%%%%%%%%%%%%%%%%%%%%%%%%%%%%%%%
%
% Known Issues:
% 1. Overflows onto second page if any column's contents are more than the
% vertical limit
% 2. Hacky space on the first bullet point on the second column.
%
%%%%%%%%%%%%%%%%%%%%%%%%%%%%%%%%%%%%%%


\documentclass[]{deedy-resume-openfont}
\usepackage{fancyhdr}

\pagestyle{fancy}
\fancyhf{}

\begin{document}

%%%%%%%%%%%%%%%%%%%%%%%%%%%%%%%%%%%%%%
%
%     LAST UPDATED DATE
%
%%%%%%%%%%%%%%%%%%%%%%%%%%%%%%%%%%%%%%

\namesection{陈}{志 凌}{ \urlstyle{same}{1262098118@qq.com} | 1358 0456 392\\
{寻找 2019 SDE/SRE 全职工作}
}

%%%%%%%%%%%%%%%%%%%%%%%%%%%%%%%%%%%%%%
%
%     COLUMN ONE
%
%%%%%%%%%%%%%%%%%%%%%%%%%%%%%%%%%%%%%%

\begin{minipage}[t]{0.3\textwidth}

%%%%%%%%%%%%%%%%%%%%%%%%%%%%%%%%%%%%%%
%     EDUCATION
%%%%%%%%%%%%%%%%%%%%%%%%%%%%%%%%%%%%%%

\section{教育经历}
\sectionsep

\subsection{华南理工大学}
\descript{工学硕士学位,计算机科学与技术}
\location{2016.09-2019.06}
\sectionsep

\subsection{华南理工大学}
\descript{工学学士学位,计算机软件}
\descript{绩点排名 6/56}
\location{2012.09-2016.06}
\sectionsep

%%%%%%%%%%%%%%%%%%%%%%%%%%%%%%%%%%%%%%
%     LINKS
%%%%%%%%%%%%%%%%%%%%%%%%%%%%%%%%%%%%%%

%%%%%%%%%%%%%%%%%%%%%%%%%%%%%%%%%%%%%%
%     SKILLS
%%%%%%%%%%%%%%%%%%%%%%%%%%%%%%%%%%%%%%

\section{技能}
\sectionsep
\subsection{编程}
\location{超过 5000 行}
Java \\
\location{1000 - 5000 行}
Go \textbullet{} Python \textbullet{} HTML \textbullet{} JavaScript \\
\location{小于 1000 行}
Scala \textbullet{} Shell \textbullet{} Solidity \\
\sectionsep

\subsection{云计算}
\location{熟悉}
Kubeflow \textbullet{} Arena  \\
\location{一般}
OpenStack \textbullet{} Docker \textbullet{} Kubernetes \\
\location{了解}
KVM \textbullet{} Ceph \textbullet{} Libvirt \\
\sectionsep

\subsection{Blockchain}
\location{一般}
ETH \textbullet{} NEBULAS \textbullet{} EOS \textbullet{} Hyperledger-Fabric \textbullet{} ABE \textbullet{} PBFT \textbullet{} RAFT \textbullet{} POW \textbullet{} POS \textbullet{} DPOS \\
\sectionsep

\subsection{Design}
\location{熟悉}
Photoshop \\
\location{一般}
Premiere \textbullet{} Aftereffect \\
\sectionsep

\subsection{后端}
\location{一般}
Spring \textbullet{} Django \textbullet{} go-restful \\
\sectionsep

\subsection{其他}
\location{一般}
开源项目管理 \textbullet{} Language Server Protocol \textbullet{} Spider \\
\sectionsep

\end{minipage}
\hfill
\begin{minipage}[t]{0.68\textwidth}

%%%%%%%%%%%%%%%%%%%%%%%%%%%%%%%%%%%%%%
%
%     COLUMN TWO
%
%%%%%%%%%%%%%%%%%%%%%%%%%%%%%%%%%%%%%%

\end{minipage}
\hfill
\begin{minipage}[t]{0.68\textwidth}

%%%%%%%%%%%%%%%%%%%%%%%%%%%%%%%%%%%%%%
%     EXPERIENCE
%%%%%%%%%%%%%%%%%%%%%%%%%%%%%%%%%%%%%%

\section{实习经历}

\sectionsep
\runsubsection{网易游戏GDC}
	\descript{基础架构开发工程师(实习)}
\location{2018.07 - 2018.09 | 广州}
\vspace{\topsep}
\begin{tightemize}
    \item 基于Kubernetes-CRD机制和TensorFlow框架实现深度学习任务调度平台
    \item 加入深度学习平台开源项目 \href{https://github.com/kubeflow/kubeflow}{Kubeflow} 社区和 \href{https://github.com/kubeflow/arena}{Arena} 并成为其中的contributor
    \item 基于 \href{https://github.com/kubeflow/kubeflow}{Kubeflow} 的组件 \href{https://github.com/kubeflow/tf-operator}{tf-operator} 实现深度学习管理平台cronus-server、cronus-cli和cronus-Dashboard
\end{tightemize}
\sectionsep

\sectionsep
\runsubsection{勤思网络科技有限公司}
	\descript{Java研发工程师(实习)}
\location{2015.07 - 2015.09 | 广州}
\vspace{\topsep}
\begin{tightemize}
    \item 负责足球社交软件“云球”项目的Java后端开发工作
    \item 采用技术框架: Spring + Resteasy + JPA
\end{tightemize}
\sectionsep


%%%%%%%%%%%%%%%%%%%%%%%%%%%%%%%%%%%%%%
%     RESEARCH
%%%%%%%%%%%%%%%%%%%%%%%%%%%%%%%%%%%%%%

\section{项目经验}
\sectionsep


\runsubsection{VirtMon}
\descript{云监控平台}
\location{2017.05 - 2017.07}
\begin{tightemize}
	\item 采用 Zookeeper管理监控系统的配置以及协调集群监控信息采集任务
	\item 使用 SpringBoot + InfluxDB 实现 monitor-server,协调管理monitor-client上报的监控信息
	\item 使用Netty框架构建RPC通讯架构
	\end{tightemize}
\sectionsep

\runsubsection{Sprout}
\descript{私有云平台}
\location{2017.05 - 2017.07}
\begin{tightemize}
    \item 采用开源云计算解决方案 OpenStack、分布式存储方案Ceph构建弹性云平台
    \item 使用 pacemaker + corosync 构建高可用云集群
    \item 针对性能问题,采用定制PKI机制替换token令牌机制、MQ镜像分区等
    \item 采用RabbitMQ-RPC构建服务统一接入层并设计Java-SDK管理云平台服务
    \end{tightemize}
\sectionsep

\runsubsection{Storest}
\descript{基于区块链的分布式文件系统}
\location{2017.11 - 至今}
\begin{tightemize}
    \item 使用超级账本Fabric构建弱中心话分布式文件系统
    \item 采用PBFT(实用性拜占庭容错算法)共识机制对区块信息进行排序
    \item 结合Merkle树结构对文件进行分片处理,同时设计质询机制保证文件可靠性
    \item 设计并实现DCP-ABE(去中心化属性基加密)算法对文件内容进行访问控制
    \end{tightemize}
\sectionsep

\section{所获奖项}
\begin{tabular}{rll}
2018         & 荣誉证书  & 迅雷区块链大赛五十强(全球共500多支队伍) \\
2018         & 周星奖 & 公有链Nebulas“全球区块链Dapp设计大赛”周星奖(200NAS) \\
2018         & 优秀奖 & 公有链Nebulas“全球区块链Dapp设计大赛”优秀奖(600NAS) \\
\end{tabular}
\begin{tabular}{rll}
2017         & 二等奖学金 & 专业排名前{30\%} \\
2016         & 二等奖学金 & 保研奖学金 \\
2015         & 一等奖学金 & 专业排名前{20\%} \\
2014         & 二等奖学金 & 专业排名前{30\%} \\
2013         & 国家励志奖学金  & 专业排名 {3/56} \\
\end{tabular}
\sectionsep


\sectionsep\end{minipage}

\end{document}
\documentclass[]{article}
